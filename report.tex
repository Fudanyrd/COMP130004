\documentclass{article}
\usepackage{amsmath}
\usepackage{CJKutf8}
\usepackage{minted}
\usepackage{graphicx}

\title{Project Report of COMP130004}
\author{Fudanyrd}
\date{December, 2023}

\begin{document}
\begin{CJK*}{UTF8}{gbsn}
\maketitle
\par Github link to this Proj: http://github.com/Fudanyrd/COMP130004

\section{基础数据结构}

\paragraph{Matrix}
 考虑到用户之间的朋友关系是对称的,所以相应图的邻接矩阵表示是对称矩阵,可以用
上三角阵压缩存储.
\paragraph{User}
 每个用户都有属性id(身份标识码),rowNum(在邻接矩阵中存储的行号),所以用一个结构体封装.
\paragraph{HashTable}
除留余数法为散列函数,用开散列法解决冲突的散列表,关键码为用户id,可以实现在$O(1)$时间内
根据用户身份标识码查找在邻接矩阵中所在行号.
\paragraph{UserList}
建立上述散列表和一个行号为关键码的向量(由于行号不重复,无需考虑堆积),可以在$O(1)$时间内
实现行号和用户标识码的一一对应.
\paragraph{Relationship}
接口类,执行handout中提及的所有功能.

\end{CJK*}
\end{document}